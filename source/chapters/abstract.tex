%% Abstract chapter
%% author Liu Peng
In recent years, the rapid development of electronics and computer science has
enabled home networking devices to become more affordable and more powerful. 
Several widely used multimedia-streaming solutions have become available in the
market. However, as a result of their different technical designs, these
standards naturally experience serious compatibility issues. Thus, end users can
have several multimedia devices, with each one using a distinctive, unique
protocol, making it challenging or even impossible sometimes to share media
between those devices. These compatibility issues have motivated the need to
determine the technological features common to the existing
multimedia-streaming standards and to develop a more easy-to-use multimedia
home networking solution.

This thesis compares the modern solutions for multimedia home networking (MHN),
including AirPlay, Miracast, Chromecast, and especially the Digital Living
Network Alliance (DLNA) standard due to its wide adoption. By conducting
research on the features and capabilities of these existing solutions, a
suitable mobile solution for MHN, which takes advantage of AirPlay, Discovery
and Launch (DIAL), and DLNA, is proposed for the Android platform. The
corresponding system architectures, features, and analysis methodologies are
also analyzed to demonstrate the competitiveness of this application.

In terms of practical contribution, an online channel proxy was integrated to
the application to fulfill the target of streaming online channels, such as
YouTube. By implementing this online channel proxy, home networking and
Internet resources can be effectively connected.

Since its first release on the Google Play Store, the application received over
one million downloads from 225 countries. According to the statistics, this
solution has proved to be competitive and successful. In addition, this thesis
discusses possible further development of this solution, and the future trends
of multimedia home networking.
