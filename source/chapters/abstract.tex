%% Abstract chapter
%% author Liu Peng
This thesis studies the modern solutions for multimedia home networking(MHN), especially the popular ones follow the Digital Living
 Network Alliance standard. By conducting a research on the features and implementations of these existing solutions, our team developed a suitable mobile solution for MHN
 which takes advantage of AirPlay, DIAL and DLNA on the Android
 platform.

Firstly, the thesis provides an overview of the popular streaming technologies, including AirPlay,
DLNA, Miracast and Chromecast. By analyzing the features and capabilities of
these streaming technologies, a universal solution is proposed for MHN in the hope of  supporting multiple protocols and bridging
 different platforms.

Secondly, different multimedia solutions are tested and a mobile Application
for home networking on Android is implemented. The corresponding system architectures, features and analysis methodologies are also discussed.

In terms of practical contribution, an online channel proxy is made in our "Streambels" application to fulfill our target of streaming online channels such as YouTube. By implementing this online channel proxy, home
 networking and Internet resources are effectively bridged together. 

Based on this thesis study, an Android application for Tuxera Inc has been published. Over the 16 months after launching this application on Google Play Store, we have been able to generate a statistic report on how users utilize this app. By making analysis on the collected statistics, a short summary of the user behavior is presented and some recommendations are given to help improve the user experience.

Lastly, a discussion on the possible further
 development of multimedia home networking is conducted to conclude this thesis study. 