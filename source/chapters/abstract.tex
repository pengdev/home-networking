%% Abstract chapter
%% author Liu Peng

 This thesis analyzes the solutions to home multimedia networking, compares 
 nowadays-popular home networking solutions, especially for the Digital Living
 Network Alliance standard. By comparing different features and
 implementations, our team developed a suitable mobile solution for multimedia
 home networking which takes advantage of both Airplay,  DIAL and DLNA
 protocols on Android platform.

 It started with a comparison of popular streaming technologies, Airplay, 
 DLNA, Miracast and Chromecast. By analyzing the features and capabilities 
 of those streaming technologies, we proposed a universal solution for 
 multimedia home networking by supporting multiple protocols and building
 bridge to different platforms.

 In the middle of the thesis, I tested different multimedia solutions and 
 we implemented a mobile Application for home networking on Android. The
 architecture, features and analyze methodology are discussed in the paper. We
 also investigate how to bridge home networking and Internet resources, an
 online channel proxy is made in our app "Streambels" to stream online
 channels like YouTube. Finally we got a published Android application for
 Tuxera Inc. The application is already published in Google Play Store for 6
 months, which generate a statistic report of how users use our app, so there
 is a short description of the user behavior, and how we could improve user
 experience by analyzing these data.

 At the end of the paper there is a discussion on the possible further
 development and the future of Home network.